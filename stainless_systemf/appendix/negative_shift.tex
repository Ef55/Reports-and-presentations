One interesting question is ``when is (or should) a negative shift be valid?''.
The first condition we came with is presented in \cref{fig:synt_shift}.

The only thing which is checked is that, after performing the shift,
the resulting term is a valid lambda term;
this simply means that no variable ends up with a negative index.

This way of thinking is perfectly valid, however it has two main downsides:
\begin{enumerate}
    \item There is no (simple) relationship between allowing a negative shift and (not) having free variables;
    \item The obtained term might have (semantically) nothing in common with the original term.
\end{enumerate}
The second point is better explained with an example:
\[\shift{-1}{0}{\abs{T}{1}} = \abs{T}{0}\]
the free variable in the body became bound after the shift.
This suggests that this precondition is too weak and allows shifts which will never be of any use,
as they ``break the semantics'' of the shifted term.

It is possible to rewrite the precondition while keeping our example in mind;
this results in \cref{fig:sem_shift}

We see that the shift presented before would be rejected, as $1+(-1) < 1$.
This new version loses some properties of the first one, 
however those properties were of no use for our implementation.
It also becomes possible to prove a really handy propriety,
which is that $\forall t \in\lterms.\ \forall d > 0.\ \forall c \in \N$
\begin{gather*}
    \inlcode{t.negShiftValidity(-d, c)}\\
    ==\\
    \inlcode{!t.hasFreeVariablesIn(c, d)}
\end{gather*}
In fact, this property holds trivially: 
the two definition are the same up to some mild changes in the \inlcode{Var} case induced by the inverted sign.

We thus decided to completely get rid of the concept of \inlcode{negShiftValidity},
and simply used \inlcode{hasFreeVariablesIn} as a precondition for negative shifts.
We still considered this reflection meaningful, as the simple precondition is unrelated to free variables, and the more thoughtful one can seem uselessly restrictive.