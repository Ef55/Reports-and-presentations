In System F, using De Brujin's indices is way more subtle than in STLC, as integers are used to represent both term and type variables. 

As mentioned, the context always allows to distinguish which is meant. However, this also means that a same integers can, at one place in the term, represent a term variable and, in an other place, represent a type variable even though they do not have anything in common (which can lead to confusions).
For term variables, everything stays the same, meaning that the first integers represent variables bound by abstractions and the other ones represent free variables.
Regarding type variables, it is quite similar, but they can either be bound by type abstractions (at the term level), or by universal types (at the type level). 

For instance, in the type $\universaltype{\arrowtype{0}{1}}$, $0$ is a bound variable whereas $1$ is a free variable. 
Similarly, in the term $\tabs{\abs{0}{2}}$, $0$ is the type variable bound by the type abstraction. Things become more intricate when we mix many of these structures. 
In the term $\tabs{\abs{(\universaltype{(\arrowtype{0}{1})})}{2}}$, $0$ is the type variable bound by the universal quantifier and $1$ is the type variable bound by the type abstraction. 

Even worse, in the term $\tabs{\abs{(\universaltype{(\arrowtype{0}{1})})}{\app{0}{1}}}$, the first $0$ and the first $1$ are the same bound type variables as in the previous example, whereas the second $0$ is the term variable bound by the abstraction and the second $1$ is a free term variable!
