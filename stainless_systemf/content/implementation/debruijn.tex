
\newcommand{\cdrange}[0]{\srangelo{c}{c+d}}

Dealing with De Brujin's notation was unarguably the most time-consuming part of the project. In particular,
the evaluation relation requires the use of negative shifts. 
When performing negative shifts, it must be verified that no variable will end up being below 0. 
Therefore, theorems to know which variables are free (the bond between free variables and validity of negative shifts is discussed in \cref{sec:negshift}) and others to know how shifting and substituting affect free variables were required. In the rest of this section, we will focus on free variables inside types; however the same explanation holds for free term variables inside terms and free type variables inside terms as well. 

Formally, the set of free variables of a type is defined as follows:
\begin{alignat*}{2}
        &\freevar{k} &&\defeq \enumset{k} \\
        &\freevar{\inlcode{char}^{*}} &&\defeq \emptyset \\
        &\freevar{\arrowtype{T_1}{T_2}} &&\defeq \freevar{T_1} \sunion \freevar{T_2}\\
        &\freevar{\universaltype{t}} &&\defeq \setbuilder{x - 1}{x \in \left(\freevar{t} \sdiff \enumset{0}\right)} 
\end{alignat*}
However, what is really interesting is not the set of free variables of a given type, but a weaker property: whether this type contains free variables in a particular range (this is again related to the ``free variables -- negative shift'' relation). 

This property translates in checking whether the intersection of the set of free variables and the range of interest is empty; however, manipulating sets and set intersections would quickly become cumbersome.

An easier solution arose using the relation in \cref{fig:freevar_rel}, whose right hand side is either trivial (\inlcode{false} and $k \in \cdrange$) or can be computed using the relation recursively.

We were thus able to define \inlcode{hasFreeVariablesIn}, a function that recursively computes the right hand side of the relation (see \cref{fig:def_hasFreeVariablesIn}), and then prove its correctness:

\begin{figure*}
    \begin{subfigure}{\linewidth}
    For $c, d \in \mathbb{N}$
    \begin{alignat*}{2}
        &\freevar{k} \sinter \cdrange \neq \emptyset && \iff k \overset{?}{\in} \cdrange \\
        &\freevar{\inlcode{char}^{*}} \sinter \cdrange \neq \emptyset && \iff \inlcode{false} \\
        &\freevar{\arrowtype{t_1}{t_2}} \sinter \cdrange \neq \emptyset && \iff 
        \big( \freevar{t_1} \sinter \cdrange \neq \emptyset \big) \lor
        \big( \freevar{t_2} \sinter {\cdrange} \neq \emptyset \big) \\
        &\freevar{\universaltype{t}} \sinter \cdrange \neq \emptyset &&\iff \freevar{t} \sinter [c + 1, c + d + 1[ \neq \emptyset
    \end{alignat*}
    \end{subfigure}
    \caption{Free variable range relation}
    \label{fig:freevar_rel}
\end{figure*}



\begin{lemma}[\inlcode{hasFreeVariablesIn} completeness]\label{thm:freevarsin_compl}
    For all type $T \in \ltypes$ and $c,d,k \in \N$ 
    such that 
    \[k \in \inlcode{T.freeVariables} \mand k \in \srangelo{c}{c+d}\]
    we have
    $\inlcode{T.hasFreeVariablesIn(c, d)}$.
\end{lemma}

\begin{lemma}[\inlcode{hasFreeVariablesIn} soundness]\label{thm:freevarsin_sound}
    For all type $T \in \ltypes$ and $c,d \in \N$ 
    such that 
    \[\forall k \in \inlcode{T.freeVariables}:\ k \not\in \srangelo{c}{c+d}\]
    we have 
    $\neg\inlcode{T.hasFreeVariablesIn(c, d)}$.
\end{lemma}

\begin{ucorollary}
    \inlcode{hasFreeVariablesIn} is correct, i.e.
    \begin{gather*}
        \inlcode{T.hasFreeVariablesIn(c, d)} \\
        \iff \\
        \freevar{T} \sinter \cdrange \neq \emptyset
    \end{gather*}
\end{ucorollary}

\begin{figure}[H]
    \centering
    \lstinputlisting{code/hasFreeVariablesIn.scala}
    \caption{Implementation of \inlcode{hasFreeVariablesIn}}
    \label{fig:def_hasFreeVariablesIn}
\end{figure}

\noindent
Therefore, we could prove theorems for \inlcode{hasFreeVariablesIn}, which was way easier since we did not have to manipulate set intersections.

First, we proved how to manipulate ranges that do not contain free variables. In fact, if there are no free variables in $\srangelo{a}{b}$ then there are no free variables in $\srangelo{a'}{b'}$ for $a' \geq a$ and $b' \leq b$. In addition, if there are no free variables in the ranges $\srangelo{a}{b}$ and $\srangelo{b}{c}$, then there are none in $[a, c[$ as well. 

After that, we needed to track free variables after shifts or substitutions, leading to two main lemmas:

\begin{lemma}\label{thm:free_shift}
    For all type $T \in \ltypes$ and $a,a',c,d \in \N$ such that
    $T$ has no free variables in $\srangelo{a}{a'}$, we have that
    \[\tshift{d}{c}{T} \text{ has no free variables in } \begin{cases}
    \srangelo{a+d}{a'+d} &\text{if } c \leq a\\
    \srangelo{a}{a'} &\text{if } c \geq a
    \end{cases}\]
\end{lemma}

\begin{lemma}\label{thm:free_subst}
    For all types $T, S \in \ltypes$ and $c,c',d, d', j \in \N$ such that
    $S$ has no free variables in $\srangelo{c}{c'}$ and $T$ has no free variables in $\srangelo{d}{d'}$, we have that
    \[\substitution{j}{S}{T} \text{ has no free variables in } \srangelo{\max(c, d)}{\min(c', d')}\]
\end{lemma}

\noindent
These theorems were used to prove that some types will have a range without free variables, which means that a negative shift can be applied to the resulting type;
in particular:

\begin{ucorollary}
    The negative shift in the \textup{T-TApp} typing rule (\cref{fig:systemf_def}) is valid,
    i.e. $\tshift{-1}{0}{\substitution{0}{\tshift{1}{0}{T_2}}{T}}$ is a type without negative variables.
\end{ucorollary}

\noindent
As already mentioned, all of this can also be applied to term shifting \& substitution as well as for type shifting \& substitution inside of terms or environments.
In particular, the same lemmas can be proved as well.
