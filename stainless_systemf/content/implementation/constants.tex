Before expanding our work to System F,
we considered extending STLC instead.
In particular, we started adding constants to the calculus.
One simple solution to do so is to ``hard-code'' the constants in an environment,
which will be used instead of the empty environment to type top-level terms.

The example in \cref{fig:td_example} can be seen as such:
0 is the only constant of the calculus, and in particular the only value of type $T$,
and as such can be interpreted as being \inlcode{unit}.

This approach however does not work if an infinite number of constants is needed, 
e.g. to represent natural numbers.

We thus implemented constants as a new construct of the calculus 
(also available on our \href{\branchurl{constants}}{repository}),
under the constraint that constants must have a ground type 
(i.e. they cannot have a function type).
This addition was not too complicated, and did not add anything difficult (nor interesting) to the proofs.

The constraint on their type however was annoying: 
one typical set of constants which is nice to have is the set of integers $\groundtype{Nat}$; 
however, having integers without any operation on them 
(such as \inlcode{Succ}, \inlcode{Pred}, \inlcode{Add},\ldots{}) does not make them worthwhile.

Lifting the constraint yields some problems when mixed with the fixpoint construct;
assume $\typerel{\emptyset{}}{\inlcode{Succ}}{\arrowtype{\groundtype{Nat}}{\groundtype{Nat}}}$:
\[ \reduces{ \fixpoint{\inlcode{Succ}} }{ ? } \]
this term is well-typed and closed ($\typerel{\emptyset{}}{\fixpoint{\inlcode{Succ}}}{\groundtype{Nat}}$) 
but is not a value; thus progress applies, even though no reduction rule applies.
Two possible solutions we thought of are:
\begin{enumerate}
    \item Make so that constants have a variant of the function type 
    (e.g. $\typerel{\emptyset{}}{\inlcode{Succ}}{\groundtype{Nat}\leadsto{}\groundtype{Nat}}$)
    to which the fixpoint combinator cannot be applied;
    \item Add a new reduction rule:
    \[ \prftree[r]{R-CstFix}{c \in \lconstants}{\reduces{\fixpoint{c}}{\app{c}{\fixpoint{c}}}} \]
\end{enumerate}

\noindent
One other thing to consider is the semantics of the constants with function type.
Assume we have 
\[\lconstants := \enumset{\inlcode{Add},\ 0,\ 1,\ 2,\ \ldots{}}\]
and consider $\typerel{\emptyset{}}{\app{\app{\inlcode{Add}}{0}}{1}}{\groundtype{Nat}}$.
We found two possible interpretations of this situation:
\begin{enumerate}
    \item Consider $\app{\app{\inlcode{Add}}{0}}{1}$ to be a value, but we then have two values, $1$ and $\app{\app{\inlcode{Add}}{0}}{1}$, which are not syntactically equal, yet semantically equal;
    \item Add some reduction rule such that $\reduces{\app{\app{\inlcode{Add}}{0}}{1}}{1}$.
\end{enumerate}
The first solution seemed to have some unwanted properties, so we discarded it; 
the second one required a way to specify the new reduction rules.
Adding the reduction rules for this example is simple and wouldn't change the proofs too much.
However, we did not want to hard-code more constructs into the calculus:
we wanted the constants to be as generic as possible.
Allowing the addition of new reduction rules seemed like some kind of safe 
``plugins'' rather than simply adding constants.

Finally, moving from STLC to System F reduced the utility of constants.
More precisely, System F allows usage of Böhm-Berarducci encoding \cite{bbencodingArt, bbencodingBlog},
which allows to encode in a practical way many sets of constants which would need to be implemented as extensions in STLC.

Considering all this, we decided to concentrate instead on adapting the proofs to System F, and discarded the constants.

