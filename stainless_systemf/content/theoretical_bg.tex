We will now formally define System F and its main theorems.
We will recall key theorems and definitions, but this section does not aim at being exhaustive, and expects prior knowledge about System F.
Most of what we will discuss in this section can be found in \cite[Chapter~23]{tapl}.
We will however adapt everything to De Bruijn notation \cite[Chapter~6]{tapl}
\begin{figure*}
    % Don't ask me why a subfigure and a raggedright are needed; I don't know, and I don't wanna
    \begin{subfigure}{\textwidth}
    \raggedright
    \textbf{Syntax:}
\begin{align*}
    \ltypes &\gntdef
        \N \gor                             % Variable Type
        \inlcode{char}^{*} \gor             % Ground Type
        \arrowtype{\ltypes}{\ltypes} \gor   % Arrow Type
        \universaltype{\ltypes}             % Universal Type
    \\
    \lenvironments &\gntdef \slist{\ltypes, \lenvironments} \gor \emptyset
    \\
    \lterms &\gntdef 
        \N \gor                             % Var
        \abs{\ltypes}{\lterms} \gor         % Abs
        \app{\lterms}{\ \lterms} \gor       % App
        \fixpoint{\lterms} \gor             % Fix
        \tabs{\lterms} \gor                 % TAbs
        \tapp{\lterms}{\ltypes}             % TApp
\end{align*}
\textbf{Reduction:}
\begin{gather*}
    \prftree[r]{R-App1}
        {\reduces{t_1}{t_1'}}
        {\reduces{\app{t_1}{t_2}}{\app{t_1'}{t_2}}}
    \qquad
    \prftree[r]{R-App2}
        {\reduces{t_2}{t_2'}}
        {\reduces{\app{t_1}{t_2}}{\app{t_1}{t_2'}}}
    \qquad
    \prftree[r]{R-TApp}
        {\reduces{t_1}{t_1'}}
        {\reduces{\tapp{t_1}{T_2}}{\tapp{t_1'}{T_2}}}
    \qquad
    \prftree[r]{R-Fix}
        {\reduces{t}{t'}}
        {\reduces{\fixpoint{t}}{\fixpoint{t'}}}
    \\
    \prftree[r]{R-Abs}
        {\reduces{t}{t'}}
        {\reduces{\abs{T}{t}}{\abs{T}{t'}}}
    \qquad
    \prftree[r]{R-TAbs}
        {\reduces{t}{t'}}
        {\reduces{\tabs{t}}{\tabs{t'}}}
    \qquad
    \prftree[r]{R-AbsApp}
        {}
        {\reduces
            {\app{(\abs{T}{t_1})}{t_2}}
            {\shift{-1}{0}{\substitution{0}{\shift{1}{0}{t_2}}{t_1}}}
        }
    \\
    \prftree[r]{R-AbsFix}
        {\phantom{a}} % need a phantom to prevent the last lines to be too close to the previous
        {\reduces
            {\fixpoint{\abs{T}{t}}}
            {\shift{-1}{0}{\substitution{0}{\shift{1}{0}{\fixpoint{\abs{T}{t}}}}{t}}}
        }
    \qquad
    \prftree[r]{R-TAbsTApp}
        {}
        {\reduces
            {\tapp{(\tabs{t_1})}{T_2}}
            {\tshift{-1}{0}{\substitution{0}{\tshift{1}{0}{T_2}}{t_1}}}
        }
\end{gather*}
\textbf{Typing:}
\begin{align*}
    &
    \prftree[r]{T-Var}
        {\slistindexing{\Gamma}{k} = T}
        {\typerel{\Gamma}{k}{T}}
    &&
    \prftree[r]{T-Abs}
        {\typerel{\slist{T_1, \Gamma}}{t}{T_2}}
        {\typerel{\Gamma}{\abs{T_1}{t}}{\arrowtype{T_1}{T_2}}}
    &&
    \prftree[r]{T-App}
        {\typerel{\Gamma}{t_1}{\arrowtype{T_2}{T}}}
        {\typerel{\Gamma}{t_2}{T_2}}
        {\typerel{\Gamma}{\app{t_1}{t_2}}{T}}
    \\
    &
    \prftree[r]{T-Fix}
        {\typerel{\Gamma}{t}{\arrowtype{T}{T}}}
        {\typerel{\Gamma}{\fixpoint{t}}{T}}
    &&
    \prftree[r]{T-TAbs}
        {\typerel{\tshift{1}{0}{\Gamma}}{t}{T}}
        {\typerel{\Gamma}{\tabs{t}}{\universaltype{T}}}
    &&
    \prftree[r]{T-TApp}
        {\typerel{\Gamma}{t}{\universaltype{T}}}
        {\typerel{\Gamma}{\tapp{t}{T_2}}{\tshift{-1}{0}{\substitution{0}{\tshift{1}{0}{T_2}}{T}}}}
\end{align*}
    \end{subfigure}
    \caption{Definition of System F}
    \label{fig:systemf_def}
\end{figure*}

System F is defined in \cref{fig:systemf_def}, and is a quite straightforward adaption from \cite[Figure 23-1]{tapl}. 
There are however a few things to note:
\begin{itemize}
    \item Integers are De Bruijn indices, and are used to represent both type and term variables; the context will always allow to distinguish which is meant. See \cref{sec:sysf_bruijn} for some examples;
    \item Ground types are represented by strings ($\inlcode{char}^{*}$);
    \item Environments are represented using lists;
    \item $\uparrow$ represents a term variable shift, whereas $\downarrow$ represents a type variable shift;
    \item In the T-TAbs typing rule, instead of adding a new dummy element\footnote{The element is usually an identifier, so adding it allows to know which identifiers are defined; using De Bruijn notation, those identifiers will always be an integer range $[0,n]$, so there is no real information to store.} to the environment, we shift the environment.
    
    The intuition behind this is discussed in \cref{sec:ttabs}.
\end{itemize}

For this definition to be complete, 
we must still define the two operations it relies on: shifting and substitution.
Note that many variants of these operations are used:
substitution of a term in a term, term shift, substitution of a type in a term,\ldots{}
since these definitions are a bit heavy, they are made available in the appendix (\cref{fig:def_shifts}).

\noindent
We can now introduce the main theorems we want to prove:
\begin{theorem}[Type judgment uniqueness]\label{thm:type_uniq}
    For all environment $\Gamma \in \lenvironments$, term $t \in \lterms$
    and types $T_1,T_2 \in \ltypes$, we have
    \[\typerel{\Gamma}{t}{T_1} \mand \typerel{\Gamma}{t}{T_2} \implies T_1 = T_2\]
\end{theorem}

\begin{theorem}[Progress]\label{thm:progress}
    For all term $t \in \lterms$ such that ${\typerel{\emptyset}{t}{T}}$ for some type $T \in \ltypes$, then either ${\exists t' \in \lterms:\ \reduces{t}{t'}}$
    or $t$ is a value.
\end{theorem}

\begin{theorem}[Preservation]\label{thm:preservation}
    For all environment $\Gamma \in \lenvironments$, terms $t,t' \in \lterms$
    and type $T \in \ltypes$, we have
    \[\typerel{\Gamma}{t}{T} \mand \reduces{t}{t'} \implies \typerel{\Gamma}{t'}{T}\]
\end{theorem}

\noindent
Theorem~\ref{thm:progress} raises two questions:
\begin{enumerate}
    \item What is a value?
    \item Does it allow a term which is both a value and reducible?
\end{enumerate}
These two questions are in fact intertwined.
The typical definition of a value is a term which is (at the top-level) an (type-)abstraction\footnote{This answer has to change when constants are added to the calculus; see \cref{sec:csts}}; in this case, the answers to the second question must be yes. We assumed this definition in our project.

It is however possible to restrict the reduction relation, in particular to make it deterministic; one such restriction is the call-by-value reduction strategy \cite[Chapter 5]{tapl}. It restricts the reduction relation in the following way:
\begin{itemize}
    \item R-Abs and R-Tabs are not allowed;
    \item R-App2 requires $t_1$ to be a value;
    \item R-AbsApp requires $t_2$ to be value ($t_1$ is also a value!).
\end{itemize}
With these restrictions, the second answer can be no.

In this project, we will prove \cref{thm:type_uniq} and \cref{thm:preservation}
in the most general way, 
i.e. for full $\beta$-reduction,
but will only prove \cref{thm:progress} for the call-by-value reduction strategy.