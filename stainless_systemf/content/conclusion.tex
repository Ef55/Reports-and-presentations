\subsection{Rundown of our work}
In this project, we:
\begin{itemize}
    \item Implemented the basis of STLC (terms, types, reduction relation,\ldots);
    \item Implemented some functions performing key operations on terms (\inlcode{reduceCallByValue}, \inlcode{deriveType});
    \item Proved that these functions are correct;
    \item Proved the main theorems of this calculus;
    \item Expanded all of this to System F.
\end{itemize}
All theorems we mentioned in the present document were proved; 
\cref{tab:proofs} lists all theorems, and indicates the name of the function in Stainless as well as the file it is implemented in.
\begin{table}[ht]
    \begin{minipage}{\linewidth} % Minipage for footnote
        \centering
        \begin{tabular}{c|l}
            Thm  & Function Name \\ \hline\hline
            \multicolumn{2}{l}{In file \pvfile{SystemF}} \\ \hline
            \ref{thm:freevarsin_compl} & \inlcode{hasFreeVariablesInCompleteness} \\
            \ref{thm:freevarsin_sound} & \inlcode{hasFreeVariablesInSoundness} \\ \hline\hline
            \multicolumn{2}{l}{In file \pvfile{Transformations}} \\ \hline
            \ref{thm:free_shift} & \inlcode{boundRangeShiftCutoff}\\
            & \inlcode{boundRangeShiftBelowCutoff} \\
            \ref{thm:free_subst} & \inlcode{boundRangeSubstitutionLemma}\textit{[1-4]} \\ \hline\hline
            \multicolumn{2}{l}{In file \pvfile{Reduction}} \\ \hline
            \ref{thm:cbv_sound} & \inlcode{reduceCallByValueSoundness}\\ \hline\hline
            \multicolumn{2}{l}{In file \pvfile{Typing}}  \\ \hline
            \ref{thm:type_uniq} & \inlcode{typeDerivationsUniqueness} \\
            \ref{thm:progress} & \inlcode{callByValueProgress}\footnote{As the name suggests, we proved this theorem for the call-by-value reduction strategy, but not for full $\beta$-reduction.} \\
            \ref{thm:preservation} & \inlcode{preservation} \\
            \ref{thm:td_sound} & \inlcode{deriveTypeSoundness} \\
            \ref{thm:td_compl} & \inlcode{deriveTypeCompleteness} \\
            \ref{thm:subst_presevation} & \inlcode{preservationUnderSubst} \\ 
            \ref{thm:types_shift} & \inlcode{shiftAllTypes} \\ \hline \hline
        \end{tabular}
    \end{minipage}
    \caption{Proofs of mentioned theorems and lemmas}
    \label{tab:proofs}
\end{table}

\subsection{Improvements and extensions}

This project could obviously be expanded by making the calculus richer (e.g. add records, mutability,\ldots) or adding typing features (existential types, System $\mathrm{F_\omega}$, \ldots).
We will however briefly discuss less obvious extensions:
\begin{itemize}
    \item Generalize \cref{thm:progress} to full $\beta$-reduction;
    \item System F has one more\footnote{In fact, it has yet one more: \emph{termination}, which is lost when the fixpoint combinator is added.} key property we didn't prove: \emph{confluence} \cite{trallthat}. 
    This property is however classically proved (as in \cite{confluenceProof}) using theorems on the transitive closure of the reduction relation and is non-constructive, which are two things which do not go well with Stainless. 
    Thus, a new (or less common) proof should be used.
    \item Implement more efficient evaluation schemes.
    Our implementation was not done with efficiency in mind, and is thus excruciatingly slow.
    Now that our implementation is done and verified, it could be used as a specification for faster implementations.
    
    One could keep the same evaluation scheme, but e.g. use a Zipper~\cite{huetZipper} or do the substitutions lazily to improve performances.
    More involved evaluation methods using e.g. interaction nets~\cite{interactionNet} or abstract machines~\cite{abstractMachines} would be more interesting, but most likely also more difficult to prove correct as well.
\end{itemize}