\subsection{Preliminaries}
Our project has been implemented and verified using Stainless~\cite{stainless} and is available on \href{\repourl}{Github}.
The code is organized as follows:
\begin{itemize}%[topsep=0.2em]
    \item \pvfile{Utils} Additional theorems on \inlcode{Option} and \inlcode{List};
    \item \pvfile{SystemF} Definition of types, terms, free variables and theorems on free variables;
    \item \pvfile{Transformations} Definitions and theorems related to substitution and shifting;
    \item \pvfile{Reduction} Definition and theorems related to the reduction relation;
    \item \pvfile{Typing} Definition of typing derivation and main theorems of System F i.e. type uniqueness and type safety.
\end{itemize}
As mentioned, the proofs were originally done for the simply typed lambda calculus (with fixpoint).
These original proofs are also available on our \href{\branchurl{stlc}}{repository}. 
Note however that many improvements were made to the proofs once we started working on System F,
and these improvements were not back-ported to the simply typed lambda calculus proofs.

Although our proofs mainly mirror the ones from \cite{tapl}, 
many tricks were required to mechanize the proofs. 
We will first introduce the more general points of the original implementation for the simply typed lambda calculus,
and then dwell on the additions we did once this original part was done.

\subsection{Typing derivations}
To prove anything related to the type of a term, we first need to define a function which tells us whether $\typerel{\Gamma}{t}{T}$. Since terms have a unique type in System F, having a function \inlcode{typeof} which, given an environment and a term, outputs its type would be convenient. However, doing so raises several issues:
\begin{itemize}
    \item Some terms are ill-typed; what should the function output in this case?
    \item Type uniqueness of System F is a property we would like to prove, but we already use it implicitly in our definition of \inlcode{typeof}: this is a chicken-and-egg problem;
    \item The type of a term depends directly of the type of the subterms and which typing rule is being applied, which are things that could be useful for some proofs. Instead, here the typing is a black box;
    \item How can we make sure that the implementation of such a procedure is correct?
\end{itemize}
To overcome all of these issues, we decided to first work on typing derivations.
A typing derivation is a proof that some term has a type, given an environment. Formally it is defined as a sequence of typing rules where each premise of a rule is either the conclusion of a previous typing rule, or an axiom. 
It has therefor a tree-like structure (as seen in \cref{fig:td_example}), where each node is a typing rule.
\begin{figure}[H]%
    \centering
    \[
    \prftree[r]{T-App}
        {\prftree[r]{T-Abs}
            {\prftree[r]{T-Var}
                {\slistindexing{\slist{T,T}}{0} = T}
                {\typerel{\slist{T,T}}{0}{T}}}
            {\typerel{\slist{T}}{\abs{T}{0}}{\arrowtype{T}{T}}}}
        {\prftree[r]{T-Var}
            {\prfassumption{\slistindexing{\slist{T}}{0} = T}}
            {\typerel{\slist{T}}{0}{T}}}
        {\typerel{\slist{T}}{\app{(\abs{T}{0})}{0}}{T}}
\]
    \caption{Example of a typing derivation}
    \label{fig:td_example}
\end{figure}

\noindent
We call the conclusion $\typerel{\Gamma}{t}{T}$ of the final typing rule the conclusion of the typing derivation, and call $\Gamma$ its environment, $t$ its term and $T$ its type.

    
This definition is easily captured by Scala's case classes. 
The trait \inlcode{TypingDerivation} represents a generic typing derivation,
and each typing rule has itw own case class (e.g. \inlcode{AppDerivation} for T-App).

However, a \inlcode{TypingDerivation} instance might not represent a correct typing derivation. Indeed, we could easily build a \inlcode{TypingDerivation} object whose subderivations do not conclude the required premises of this derivation.
%terms are not the subterms of the original term. %A reformuler
We thus need a function, \inlcode{isValid}, which checks whether a given typing derivation is correct. More formally, if $\sigma$ is a typing derivation concluding $\typerel{\Gamma}{t}{T}$ whose subderivations are $\{\sigma_i\}$, then \inlcode{isValid} is true if:
\begin{itemize}
    \item \inlcode{isValid} is true for each $\sigma_i$;
    \item The environment and type of each $\sigma_i$ must correctly relate with $T$ and $\Gamma$;
    the exact relation depends on the typing rule being used
    (e.g. for most rules, $\sigma_i$'s environment must be equal to $\Gamma$,
    but in the case of T-TAbs it must be equal to $\tshift{1}{0}{\Gamma}$);
    \item The term of each $\sigma_i$ is the correct subterm of $t$.
\end{itemize}
We were then able to implement a function \inlcode{deriveType}, that given a term and an environment produces a typing derivation.

Now that all this machinery is settled, we can find solutions to all of our issues:
\begin{itemize}
    \item If a term $t$ is ill-typed then for all typing derivations with term $t$, \inlcode{isValid} will be false, and \inlcode{deriveType} can simply return \inlcode{None};
    \item We can prove that two valid typing derivations with the same term and environment are equal; in particular, they conclude the same type. 
    This proves \cref{thm:type_uniq}, without using it as an assumption!
    \item We can track all the typing rules that have been used, and we know the type of all the subterms thanks to the generated typing derivation.
    \item Finally, we can prove the correctness of \inlcode{deriveType}
    by proving the following two lemmas:
\end{itemize} 

\begin{lemma}[\inlcode{deriveType} soundness]\label{thm:td_sound}
    For all environment $\Gamma \in \lenvironments$ and term $t \in \lterms$
    such that $\inlcode{deriveType($\Gamma$,t)}$ is defined, we have
    \[ \inlcode{deriveType($\Gamma$,t).isValid} \]
\end{lemma}

\begin{lemma}[\inlcode{deriveType} completeness]\label{thm:td_compl}
    For all valid typing derivation $\sigma$ with environment $\Gamma$ and term $t$, we have
    \[ \inlcode{deriveType($\Gamma$,t)} = \sigma \]
\end{lemma}

\noindent
It is even possible to define the \inlcode{typeof} function we originally wanted:
\begin{figure}[h]
    \centering
    \lstinputlisting{code/typeof.scala}
    \caption{Implementation of \inlcode{typeof}}
    \label{fig:def_typeof}
\end{figure}

\subsection{Reduction relation}

The way we implemented and verified the reduction relation is a bit reminiscent of how typing relations were done, but not quite similar:

\begin{itemize}
    \item The ``specification'' was first implemented as a \inlcode{reducesTo} function, which is used to determine whether a term reduces to another. It is implemented such that it is possible, without too much trouble, to ensure that it matches the definition in \cref{fig:systemf_def};
    \item We then defined a function which has to determine if a term does reduce, and if it is the case, to which other term. This function also defines a reduction strategy. We decided to use call-by-value as a strategy, so the function is \inlcode{reduceCallByValue};
    \item We can then prove progress for the chosen reduction strategy.
\end{itemize}

\begin{figure}[h]
    \centering
    \lstinputlisting[basicstyle=\fontsize{8}{8}\ttfamily]{code/reduction_signatures.scala}
    \caption{The reduction functions}
\end{figure}

\noindent
One not so obvious thing to verify before moving to progress is that the reduction strategy is sound:
\begin{theorem}[\inlcode{reduceCallByValue} soundness]\label{thm:cbv_sound}
    For all term $t \in \lterms$ such that \inlcode{reduceCallByValue(t)} is defined, we have
    \[\reduces{t}{\inlcode{reduceCallByValue(t)}}\]
\end{theorem}

\noindent
Note that there is no real ``completeness'' criterion for reduction strategies,
except for the progress property, which can be seen as a kind of completeness.

It is then (surprisingly) trivial to prove progress:
\begin{figure}[h]
    \centering
    \lstinputlisting{code/progress.scala}
    \caption{Stainless proof of \cref{thm:progress}}
\end{figure}

\noindent
The proof is however tied to one particular evaluation strategy. 
One could try to generalize as much as possible, i.e. by proving it for full $\beta$-reduction. It might even be possible to prove it for some kind of non-deterministic reduction strategy\footnote{Either by having the relation being truly non-deterministic, as Stainless seems to have a \href{https://github.com/epfl-lara/stainless/blob/160a14a5bec19b6fc386b839edbed07cdd3e70fd/frontends/library/stainless/util/Random.scala}{randomness} source, or by having the function return all reduction candidates.}, but the proof cannot be given for all evaluation strategies at once. In particular, any reduction strategy which disallows a computation rule, such as R-AbsApp, (trivially) cannot ensure progress.

\subsection{Free variables and negative shifts}

\newcommand{\cdrange}[0]{\srangelo{c}{c+d}}

Dealing with De Brujin's notation was unarguably the most time-consuming part of the project. In particular,
the evaluation relation requires the use of negative shifts. 
When performing negative shifts, it must be verified that no variable will end up being below 0. 
Therefore, theorems to know which variables are free (the bond between free variables and validity of negative shifts is discussed in \cref{sec:negshift}) and others to know how shifting and substituting affect free variables were required. In the rest of this section, we will focus on free variables inside types; however the same explanation holds for free term variables inside terms and free type variables inside terms as well. 

Formally, the set of free variables of a type is defined as follows:
\begin{alignat*}{2}
        &\freevar{k} &&\defeq \enumset{k} \\
        &\freevar{\inlcode{char}^{*}} &&\defeq \emptyset \\
        &\freevar{\arrowtype{T_1}{T_2}} &&\defeq \freevar{T_1} \sunion \freevar{T_2}\\
        &\freevar{\universaltype{t}} &&\defeq \setbuilder{x - 1}{x \in \left(\freevar{t} \sdiff \enumset{0}\right)} 
\end{alignat*}
However, what is really interesting is not the set of free variables of a given type, but a weaker property: whether this type contains free variables in a particular range (this is again related to the ``free variables -- negative shift'' relation). 

This property translates in checking whether the intersection of the set of free variables and the range of interest is empty; however, manipulating sets and set intersections would quickly become cumbersome.

An easier solution arose using the relation in \cref{fig:freevar_rel}, whose right hand side is either trivial (\inlcode{false} and $k \in \cdrange$) or can be computed using the relation recursively.

We were thus able to define \inlcode{hasFreeVariablesIn}, a function that recursively computes the right hand side of the relation (see \cref{fig:def_hasFreeVariablesIn}), and then prove its correctness:

\begin{figure*}
    \begin{subfigure}{\linewidth}
    For $c, d \in \mathbb{N}$
    \begin{alignat*}{2}
        &\freevar{k} \sinter \cdrange \neq \emptyset && \iff k \overset{?}{\in} \cdrange \\
        &\freevar{\inlcode{char}^{*}} \sinter \cdrange \neq \emptyset && \iff \inlcode{false} \\
        &\freevar{\arrowtype{t_1}{t_2}} \sinter \cdrange \neq \emptyset && \iff 
        \big( \freevar{t_1} \sinter \cdrange \neq \emptyset \big) \lor
        \big( \freevar{t_2} \sinter {\cdrange} \neq \emptyset \big) \\
        &\freevar{\universaltype{t}} \sinter \cdrange \neq \emptyset &&\iff \freevar{t} \sinter [c + 1, c + d + 1[ \neq \emptyset
    \end{alignat*}
    \end{subfigure}
    \caption{Free variable range relation}
    \label{fig:freevar_rel}
\end{figure*}



\begin{lemma}[\inlcode{hasFreeVariablesIn} completeness]\label{thm:freevarsin_compl}
    For all type $T \in \ltypes$ and $c,d,k \in \N$ 
    such that 
    \[k \in \inlcode{T.freeVariables} \mand k \in \srangelo{c}{c+d}\]
    we have
    $\inlcode{T.hasFreeVariablesIn(c, d)}$.
\end{lemma}

\begin{lemma}[\inlcode{hasFreeVariablesIn} soundness]\label{thm:freevarsin_sound}
    For all type $T \in \ltypes$ and $c,d \in \N$ 
    such that 
    \[\forall k \in \inlcode{T.freeVariables}:\ k \not\in \srangelo{c}{c+d}\]
    we have 
    $\neg\inlcode{T.hasFreeVariablesIn(c, d)}$.
\end{lemma}

\begin{ucorollary}
    \inlcode{hasFreeVariablesIn} is correct, i.e.
    \begin{gather*}
        \inlcode{T.hasFreeVariablesIn(c, d)} \\
        \iff \\
        \freevar{T} \sinter \cdrange \neq \emptyset
    \end{gather*}
\end{ucorollary}

\begin{figure}[H]
    \centering
    \lstinputlisting{code/hasFreeVariablesIn.scala}
    \caption{Implementation of \inlcode{hasFreeVariablesIn}}
    \label{fig:def_hasFreeVariablesIn}
\end{figure}

\noindent
Therefore, we could prove theorems for \inlcode{hasFreeVariablesIn}, which was way easier since we did not have to manipulate set intersections.

First, we proved how to manipulate ranges that do not contain free variables. In fact, if there are no free variables in $\srangelo{a}{b}$ then there are no free variables in $\srangelo{a'}{b'}$ for $a' \geq a$ and $b' \leq b$. In addition, if there are no free variables in the ranges $\srangelo{a}{b}$ and $\srangelo{b}{c}$, then there are none in $[a, c[$ as well. 

After that, we needed to track free variables after shifts or substitutions, leading to two main lemmas:

\begin{lemma}\label{thm:free_shift}
    For all type $T \in \ltypes$ and $a,a',c,d \in \N$ such that
    $T$ has no free variables in $\srangelo{a}{a'}$, we have that
    \[\tshift{d}{c}{T} \text{ has no free variables in } \begin{cases}
    \srangelo{a+d}{a'+d} &\text{if } c \leq a\\
    \srangelo{a}{a'} &\text{if } c \geq a
    \end{cases}\]
\end{lemma}

\begin{lemma}\label{thm:free_subst}
    For all types $T, S \in \ltypes$ and $c,c',d, d', j \in \N$ such that
    $S$ has no free variables in $\srangelo{c}{c'}$ and $T$ has no free variables in $\srangelo{d}{d'}$, we have that
    \[\substitution{j}{S}{T} \text{ has no free variables in } \srangelo{\max(c, d)}{\min(c', d')}\]
\end{lemma}

\noindent
These theorems were used to prove that some types will have a range without free variables, which means that a negative shift can be applied to the resulting type;
in particular:

\begin{ucorollary}
    The negative shift in the \textup{T-TApp} typing rule (\cref{fig:systemf_def}) is valid,
    i.e. $\tshift{-1}{0}{\substitution{0}{\tshift{1}{0}{T_2}}{T}}$ is a type without negative variables.
\end{ucorollary}

\noindent
As already mentioned, all of this can also be applied to term shifting \& substitution as well as for type shifting \& substitution inside of terms or environments.
In particular, the same lemmas can be proved as well.


\subsection{Constants}\label{sec:csts}
Before expanding our work to System F,
we considered extending STLC instead.
In particular, we started adding constants to the calculus.
One simple solution to do so is to ``hard-code'' the constants in an environment,
which will be used instead of the empty environment to type top-level terms.

The example in \cref{fig:td_example} can be seen as such:
0 is the only constant of the calculus, and in particular the only value of type $T$,
and as such can be interpreted as being \inlcode{unit}.

This approach however does not work if an infinite number of constants is needed, 
e.g. to represent natural numbers.

We thus implemented constants as a new construct of the calculus 
(also available on our \href{\branchurl{constants}}{repository}),
under the constraint that constants must have a ground type 
(i.e. they cannot have a function type).
This addition was not too complicated, and did not add anything difficult (nor interesting) to the proofs.

The constraint on their type however was annoying: 
one typical set of constants which is nice to have is the set of integers $\groundtype{Nat}$; 
however, having integers without any operation on them 
(such as \inlcode{Succ}, \inlcode{Pred}, \inlcode{Add},\ldots{}) does not make them worthwhile.

Lifting the constraint yields some problems when mixed with the fixpoint construct;
assume $\typerel{\emptyset{}}{\inlcode{Succ}}{\arrowtype{\groundtype{Nat}}{\groundtype{Nat}}}$:
\[ \reduces{ \fixpoint{\inlcode{Succ}} }{ ? } \]
this term is well-typed and closed ($\typerel{\emptyset{}}{\fixpoint{\inlcode{Succ}}}{\groundtype{Nat}}$) 
but is not a value; thus progress applies, even though no reduction rule applies.
Two possible solutions we thought of are:
\begin{enumerate}
    \item Make so that constants have a variant of the function type 
    (e.g. $\typerel{\emptyset{}}{\inlcode{Succ}}{\groundtype{Nat}\leadsto{}\groundtype{Nat}}$)
    to which the fixpoint combinator cannot be applied;
    \item Add a new reduction rule:
    \[ \prftree[r]{R-CstFix}{c \in \lconstants}{\reduces{\fixpoint{c}}{\app{c}{\fixpoint{c}}}} \]
\end{enumerate}

\noindent
One other thing to consider is the semantics of the constants with function type.
Assume we have 
\[\lconstants := \enumset{\inlcode{Add},\ 0,\ 1,\ 2,\ \ldots{}}\]
and consider $\typerel{\emptyset{}}{\app{\app{\inlcode{Add}}{0}}{1}}{\groundtype{Nat}}$.
We found two possible interpretations of this situation:
\begin{enumerate}
    \item Consider $\app{\app{\inlcode{Add}}{0}}{1}$ to be a value, but we then have two values, $1$ and $\app{\app{\inlcode{Add}}{0}}{1}$, which are not syntactically equal, yet semantically equal;
    \item Add some reduction rule such that $\reduces{\app{\app{\inlcode{Add}}{0}}{1}}{1}$.
\end{enumerate}
The first solution seemed to have some unwanted properties, so we discarded it; 
the second one required a way to specify the new reduction rules.
Adding the reduction rules for this example is simple and wouldn't change the proofs too much.
However, we did not want to hard-code more constructs into the calculus:
we wanted the constants to be as generic as possible.
Allowing the addition of new reduction rules seemed like some kind of safe 
``plugins'' rather than simply adding constants.

Finally, moving from STLC to System F reduced the utility of constants.
More precisely, System F allows usage of Böhm-Berarducci encoding \cite{bbencodingArt, bbencodingBlog},
which allows to encode in a practical way many sets of constants which would need to be implemented as extensions in STLC.

Considering all this, we decided to concentrate instead on adapting the proofs to System F, and discarded the constants.



\subsection{From STLC to System F}
The main challenges when moving to System F are the two new typing rules T-TAbs and T-TApp.
In particular, the fact that the shifting and substitution operations may appear in typing derivations means that computing the type of a term becomes non-trivial, 
e.g. assuming $\typerel{\emptyset}{t}{T}$, we get
\begin{gather*}
    \typerel{\emptyset}{\tapp{\tapp{(\tabs{\tabs{t}})}{S_1}}{S_2}}{T'} \qquad \textup{where}\\
    T' \defeq \tshift{-1}{0}{\substitution{0}{\tshift{-1}{0}{S_2}}{
        \tshift{-1}{0}{\substitution{0}{\tshift{-1}{0}{S_1}}{T}}
    }}
\end{gather*}

\noindent
At first, we proved some theorems depending on our exact needs at the time,
but ended up realizing that some general commutativity theorems on shifts and substitutions would be more efficient, as all computations are combinations of the two at low-level.

However, the commutativity of these operations is more convoluted than one may expect;
more precisely, they highly depend on the ordering relation between the substituted index, the shift cutoff and the shift distance.

\subsubsection{Proving through rewriting}
When looking at the reduction rules with type preservation in mind,
it seemed quite natural for us to prove statements such as:
\begin{lemma}\label{thm:subst_presevation}
    For all environment $\Gamma \in \lenvironments$, terms $t,s \in \lterms$, types $T,S \in \ltypes$ and integer $j \in \N$ such that 
    \[
        \typerel{\Gamma}{t}{T} \mand 
        \typerel{\Gamma}{s}{S} \mand
        \slistindexing{\Gamma}{j} = S
    \]
    we have
    \[ \typerel{\Gamma}{\substitution{j}{s}{t}}{T} \]
\end{lemma}

Such statements can be proved by ``rewriting'' the typing derivation.
This proceeds by induction on the original typing derivation, and modifies its subderivations as well as its conclusion so that the conclusion of the typing derivation becomes the expected typing relation (e.g. ${\typerel{\Gamma}{\substitution{j}{s}{t}}{T}}$), while keeping the whole derivation valid.

This approach is quite powerful, and allows to prove some less obvious properties such as the following:
\begin{lemma}\label{thm:types_shift}
    For all environment $\Gamma \in \lenvironments$, term $t \in \lterms$ and type $T \in \ltypes$, we have
    \[\typerel{\Gamma}{t}{T} \implies \typerel{\tshift{s}{c}{\Gamma}}{\tshift{s}{c}{t}}{\tshift{s}{c}{T}}\]
\end{lemma}

\noindent
By using this approach to prove how typing changed after one single operation,
and chaining many of them, we were able to prove preservation (\cref{thm:preservation}) for System F.